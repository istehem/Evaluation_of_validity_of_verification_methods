
\section{State Machine}
The watchdog manager has a state machine \ref{FIG:GLOBALSTATUSES},
which which transitions depend on the on global variables, changes of global
variables, and the current state. If the behavior of the watchdog manager is
correct and the manager is actually running, or in other words not deactivated,
it stays in 'WDGM\_GLOBAL\_STATUS\_OK'. There are however lots of reasons for
that the status will change from the correct state. It depends on arguments to
the API calls but also in which order commands are called. Also the supplied
AUTOASAR configuration is relevant, because it specifies tolerances of faulty
behavior, it may indirect disable some states and state transition or make some
transition more likely to happen. The indirect affect can for instance come from
the number of checkpoints supplied in the configuration. A correct behavior of
the watchdog manager depends on that checkpoints are reached and does so in the
right order.

The only function that can change the global status is the mainfunction. The
main function is continuously called in a given time interval, note that the
timing is not used when using quickcheck \ref{SEC:CALLING_COMMANDS}, schedule be
the run time environment (RTE).

\begin{figure}[h!]
\label{FIG:GLOBALSTATUSES}
\caption{State diagram that shows possible trasitions between states}
\begin{center}
\includegraphics{pictures/globalstatuses.jpg}
\end{center}
\end{figure}



\section{Plots}

\subsection{BSI}
\begin{figure}[h!]
\label{FIG:BSI}
\caption{BSI configuration}
\begin{center}
\subfigure[Shows percentage of each possible command executed]{
\label{FIG:COMMANDS_BSI}
\includegraphics{generated_pictures/history_commands_bsi.pdf}
}

\subfigure[Shows percentage of each possible global status hit]{
\label{FIG:STATUSES_BSI}
\includegraphics{generated_pictures/history_statuses_bsi.pdf}
}
\end{center}
\end{figure}

\begin{table}[h!]
\label{TABLE:STATUSES_BSI}

    \begin{tabular}{r|ccccc}
        (From,To)   & DEACTIVATED & EXPIRED & FAILED & OK & STOPPED \\
        \hline
        DEACTIVATED &  2.13\% &  0.00\% &  0.00\% &  9.03\% &  0.00\% \\
        EXPIRED     &  0.00\% &  0.00\% &  0.00\% &  0.00\% &  0.00\% \\
        FAILED      &  0.00\% &  0.00\% &  0.00\% &  0.00\% &  0.00\% \\
        OK          &  3.26\% &  0.00\% &  0.00\% & 85.58\% &  0.00\% \\
        STOPPED     &  0.00\% &  0.00\% &  0.00\% &  0.00\% &  0.00\%
      \end{tabular}
    

\caption{bsi configuration}
\end{table}

\subsection{freescale}

\begin{figure}[h!]
\label{FIG:FREESCALE}
\caption{Freescale configuration}
\begin{center}

\subfigure[Shows percentage of each possible command executed]{
\includegraphics{generated_pictures/history_commands_freescale.pdf}
\label{FIG:COMMANDS_FREESCALE}
}
\subfigure[Shows percentage of each possible global status hit]{
\label{FIG:STATUSES_FREESCALE}
\includegraphics{generated_pictures/history_statuses_freescale.pdf}
}
\end{center}
\end{figure}

\begin{table}[!h]
\label{TABLE:STATUSES_FREESCALE}

    \begin{tabular}{r|ccccc}
        \backslashbox{From}{To}
                    & DEACTIVATED & EXPIRED & FAILED & OK & STOPPED \\
        \hline
        DEACTIVATED & \bf{08.52}\% & 00.00\%       & 00.00\%       & \bf{06.91}\% & 00.00\% \\
        EXPIRED     & 00.00\%       & \bf{04.84}\% & 00.00\%       & 00.00\%       & \bf{00.36}\% \\
        FAILED      & 00.00\%       & \bf{00.06}\% & \bf{04.04}\% & \bf{00.08}\% & \bf{00.07}\% \\
        OK          & \bf{02.78}\% & \bf{00.38}\% & \bf{00.49}\% & \bf{51.58}\% & \bf{00.05}\% \\
        STOPPED     & 00.00\%       & 00.00\%       & 00.00\%       & 00.00\%       & \bf{19.82}\%
      \end{tabular}
    

\caption{freescale configuration}
\end{table}

\subsection{example}

\begin{figure}[h!]
\caption{Example configuration}
\label{FIG:EXAMPLE}
\begin{center}

\subfigure[Shows percentage of each possible command executed]{
\label{FIG:COMMANDS_EXAMPLE}
\includegraphics{generated_pictures/history_commands_example.pdf}
}

\subfigure[Shows percentage of each possible global status hit]{
\label{FIG:STATUSES_EXAMPLE}
\includegraphics{generated_pictures/history_statuses_example.pdf}
}

\end{center}
\end{figure}

\begin{table}[!h]
\label{TABLE:STATUSES_EXAMPLE}

    \begin{tabular}{r|ccccc}
        (From,To)   & DEACTIVATED & EXPIRED & FAILED & OK & STOPPED \\
        \hline
        DEACTIVATED &  2.41\% &  0.00\% &  0.00\% &  7.53\% &  0.00\% \\
        EXPIRED     &  0.00\% & 23.93\% &  0.00\% &  0.00\% &  0.64\% \\
        FAILED      &  0.00\% &  0.14\% &  3.41\% &  0.00\% &  0.00\% \\
        OK          &  1.35\% &  2.77\% &  0.21\% & 43.11\% &  0.14\% \\
        STOPPED     &  0.00\% &  0.00\% &  0.00\% &  0.00\% & 14.35\%
      \end{tabular}
    

\caption{example configuration}
\end{table}

\section{Handle bugs in the C code}
\label{sec:handlebugs}

\section{Functional safety analysis}
\subsection{Definition of time}
\label{SEC:FUNCTIONAL_SAFETY_TIME}

\section{Statistics}
